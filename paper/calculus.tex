The proposed \phic{} is based on set theory~\citep{jech2013set} and lambda calculus,
representing objects as sets of pairs and their internals as $\lambda$~terms.
The rest of the section contains formal definitions of
data, objects, attributes, abstraction, application, decoration, and dataization.

\subsection{Objects and Data}

\begin{eodefinition}\label{def:object}
An \textbf{object} is a set of ordered pairs $(a_i, v_i)$ such that
$a_i$ is an identifier, all $a_i$ are different, and $v_i$ is an object.
\end{eodefinition}

An identifier is either $\varphi$, $\rho$, or, by convention, a text without
spaces starting with a small-case English letter in typewriter font.

The object at \lref{book} may be represented as
\begin{equation}\label{eq:book}
\begin{split}
\ff{book} & = \left\{\begin{matrix*}[l]
  (\ff{isbn}, \emptyset) \\
\end{matrix*}\right\},
\end{split}
\end{equation}
where \ff{isbn} is an identifier and $\emptyset$ is an empty
set, which is a proper object, according to the Def.~\ref{def:object}.

\begin{eodefinition}\label{def:data}
An object may have properties of \textbf{data},
which is a computation platform dependable entity and is not
decomposable any further within the scope of \phic{}.
\end{eodefinition}

What exactly is data may depend on the
implementation platform, but most certainly would include
byte arrays, integers, floating-point numbers,
string literals, and boolean values.

The object at \lrefs{book2}{book2-end} may be represented as
\begin{equation}\label{eq:book2}
\begin{split}
\ff{book2} & = \left\{\begin{matrix*}[l]
  (\ff{isbn}, \emptyset) \\
  (\ff{title}, \ff{"Object Thinking"}) \\
  (\ff{price}, \ff{memory}) \\
\end{matrix*}\right\}, \\
\end{split}
\end{equation}
where \ff{isbn}, \ff{title}, and \ff{price} are identifiers,
\ff{memory} is an object defined somewhere else,
and the text in double quotes is data.

\subsection{Attributes}

\begin{eodefinition}\label{def:attribute}
In an object $x$, $a$ is a free \textbf{attribute}
iff $(a, \emptyset) \in x$; it is a bound attribute
iff $\exists (a, v)\in x$ and $v\not=\emptyset$.
\end{eodefinition}

In the Eq.~\ref{eq:book2}, identifiers \ff{isbn}, \ff{title}, and \ff{price}
are the attributes of the object \ff{book2}.
The attribute \ff{isbn} is free, while the other two are bound.

\begin{eodefinition}\label{def:dot}
If $x$ is an object and $\exists (a, v) \in x$, then $v$ may be referenced as $x.a$;
this referencing mechanism is called \textbf{dot notation}.
\end{eodefinition}

Both free and bound attributes of an object are accessible using
the dot notation. There is no such thing as
visibility restriction in \phic{}:
all attributes are visible to all objects outside of the one they belong to.

It's possible to chain attribute references using dot notation, for example
$\ff{book2}.\ff{price}.\ff{neg}$ is a valid expression, which means
``taking the attribute \ff{price} from the object \ff{book2} and then
taking the attribute \ff{neg} from it.''

\begin{eodefinition}\label{def:scope}
If $x(a_i, v_i)$ is an object, than $\hat{x}$, a set consisting of all $a_i$,
is its \textbf{scope} and the cardinality of $|\hat{x}|$ is
the \textbf{arity} of $x$.
\end{eodefinition}

For example, the scope of the object at the Eq.~\ref{eq:book2} consists of three identifiers:
\ff{isbn}, \ff{title}, and \ff{price}.

\subsection{Abstraction}

\begin{eodefinition}\label{def:abstraction}
An object $x$ is \textbf{abstract} iff at least one of its attributes is free,
i.e. $\exists (a, \emptyset)\in x$;
the process of creating such an object is called \textbf{abstraction}.
\end{eodefinition}

An alternative ``arrow notation'' may be used to denote an object $x$ in a more
compact way, where free attributes stay in the parentheses on the left side of the
mapping symbol $\mapsto$ and pairs,
which represent bound attributes, stay on the right side, in double-square brackets.
The Eq.~\ref{eq:book2} may be written as
\begin{equation}\label{eq:book2-compact}
\begin{split}
& \ff{book2}(\ff{isbn}) \mapsto \llbracket \br
& \quad \ff{title} \mapsto \ff{"Object Thinking"}, \br
& \quad \ff{price} \mapsto \ff{memory} \br
& \rrbracket. \\
\end{split}
\end{equation}

\subsection{Application}

\begin{eodefinition}\label{def:application}
If $x$ and $y$ are objects, than an \textbf{application} of $y$ to $x$ is
  a \textbf{copy} of $x$, a new object that consists of pairs $(a\in\hat{x},v)$ such that
  $v=y.a$ if $x.a=\emptyset$ and $v=x.a$ otherwise.
\end{eodefinition}

Abstraction makes some free attributes of $x$ bound---by binding objects to them.
The produced object has exactly the same set of attributes, but some of them,
which were free before, become bound.

It is not expected that all free attributes turn into bound ones during application.
Some of them may remain free, which will lead
to creating a new abstract object. To the contrary,
if all free attributes are substituted with \emph{arguments} during copying,
a newly created object will be \emph{closed}.

Once set, bound attributes may not be reset
This may be interpreted as \emph{immutability} property of objects.

Arrow notation may also be used to denote object copying,
where the names of the attributes, which remain free, stay in the brackets
on the left side of the mapping symbol $\mapsto$,
while objects $P$ provided as arguments stay on the right side,
in the brackets. For example, the object at \lref{point-copy} may be written as
\begin{equation}\label{eq:point}
\begin{split}
& \ff{point}(\ff{x} \mapsto \ff{5}, \ff{y} \mapsto \ff{-3}).\ff{distance}(\br
& \quad \ff{p} \mapsto \ff{point}(\ff{x} \mapsto \ff{13}, \ff{y} \mapsto \ff{3.9}) \br
& ),
\end{split}
\end{equation}
and may further be simplified since the order of parameters is obvious:
\begin{equation}
\begin{split}
\ff{point}(\ff{5}, \ff{-3}).\ff{distance}(\ff{point}(\ff{13}, \ff{3.9})).
\end{split}
\end{equation}

\subsection{Inner and Parent Objects}

\begin{eodefinition}\label{def:parent}
If an object $x$ is bound to an attribute of
an object $y$, than $x.\rho$ denotes $y$;
the object $x$ is \textbf{inner} object, while $y$ is its \textbf{parent};
an object, which is not bound to any attributes, is called \textbf{anonymous}.
\end{eodefinition}

For example, the object at \lrefs{book3}{book3-end}
has three inner objects bound to attributes \ff{title}, \ff{price}, and \ff{set-price}:
\begin{equation}\label{eq:book3}
\begin{split}
& \ff{book3}(\ff{isbn}) \mapsto \llbracket \br
& \quad \ff{title} \mapsto \ff{"Object Thinking"}, \br
& \quad \ff{price} \mapsto \ff{memory}, \br
& \quad \ff{set-price}(\ff{p}) \mapsto \llbracket \br
& \quad \quad \varphi \mapsto \rho.\ff{price}.\ff{write}(\ff{p}) \br
& \quad \rrbracket \br
& \rrbracket, \\
\end{split}
\end{equation}
where $\rho.\ff{price}$ refers to the attribute \ff{price}
of the parent object \ff{book3}. It's not always required to mention
$\rho$ explicitly, however it it may be present for the sake of
disambiguation.

Since the same object may be bound to more then one attribute,
the parent $\rho$ may depend on where the object
is bound.

\subsection{Decoration}

\begin{eodefinition}\label{def:decorator}
If $x$ and $y$ are objects and $x.\varphi = y$, then
  $\forall a (x.a = y.a)$ if $a \not\in \hat{x}$;
  this means that $x$ is \textbf{decorating} $y$.
\end{eodefinition}

Here, $\varphi$ is a special identifier denoting the object being decorated
within the scope of the decorator.

For example, the object at \lrefs{circle}{circle-end} would
be denoted by this formula:
\begin{equation}\label{eq:c-empty}
\begin{split}
& \ff{circle}(\ff{center}, \ff{radius}) \mapsto \llbracket \br
& \quad \varphi \mapsto \ff{center}, \br
& \quad \ff{is-inside}(\ff{p}) \mapsto \llbracket \br
& \quad \quad \varphi \mapsto \rho.\varphi.\ff{distance}(\ff{p}).\ff{leq}(\ff{radius}) \br
& \quad \rrbracket \br
& \rrbracket,
\end{split}
\end{equation}
while the application of it would look like:
\begin{equation}
\begin{split}
\ff{c} & \mapsto \ff{circle}(\ff{point}(\ff{-3}, \ff{9}), \ff{40}),
\end{split}
\end{equation}
producing:
\begin{equation}\label{eq:c-fin}
\begin{split}
& \ff{c} \mapsto \llbracket \br
& \quad \ff{center} \mapsto \ff{point}(\ff{-3}, \ff{9}), \br
& \quad \ff{radius} \mapsto \ff{40}, \br
& \quad \varphi \mapsto \ff{center}, \br
& \quad \ff{is-inside}(\ff{p}) \mapsto \llbracket \br
& \quad \quad \varphi \mapsto \rho.\ff{distance}(\ff{p}).\ff{leq}(\ff{radius}) \br
& \quad \rrbracket \br
& \rrbracket.
\end{split}
\end{equation}

Because of decoration, the expression
$\rho$.$\varphi$.\ff{distance} in the Eq.~\ref{eq:c-empty} is semantically equivalent to the expression
$\rho$.\ff{distance} in the Eq.~\ref{eq:c-fin}.

The following expression makes a new object \ff{is}, which represents
a sequence of object applications ending with a copy of \ff{leq}:
\begin{equation}\label{eq:is}
\begin{split}
& \ff{is} \mapsto \ff{c}.\ff{is-inside}(\ff{point}(\ff{1}, \ff{7})),
\end{split}
\end{equation}
producing:
\begin{equation}\label{eq:c-fin2}
\begin{split}
& \ff{c} \mapsto \llbracket \br
& \quad \ff{center} \mapsto \ff{point}(\ff{-3}, \ff{9}), \br
& \quad \ff{radius} \mapsto \ff{40}, \br
& \quad \varphi \mapsto \ff{center}, \br
& \quad \ff{is-inside} \mapsto \llbracket \br
& \quad \quad \ff{p} \mapsto \ff{point}(\ff{1}, \ff{7}), \br
& \quad \quad \varphi \mapsto \rho.\ff{distance}(\ff{p}).\ff{leq}(\ff{radius}) \br
& \quad \rrbracket \br
& \rrbracket.
\end{split}
\end{equation}

\subsection{Atoms}

\begin{eodefinition}\label{def:atom}
If $\lambda s.M$ is a function of one argument $s$ returning an object,
than it's an abstract object called an \textbf{atom} and $M$ is its $\lambda$~term.
\end{eodefinition}

For example, the atom at \lref{sum-def} would be represented as
\begin{equation}
\begin{split}
& \ff{sum}(\ff{x}) \mapsto \lambda s . \sum\limits_{i=0}^{|s[0]| - 1} s[0].\ff{x}.\ff{get}(i),
\end{split}
\end{equation}
where the function calculates an arithmetic sum of all items
in the array \ff{x} and returns the result as a data. The argument of
the function is a vector $s$ where the first element is the object under
consideration, the second element is its parent object, the third element
is the parent of the parent, and so on. Thus, $s[0]$ is the object
\ff{sum} itself, while $s[0].\ff{x}$ is its inner object \ff{x},
and $s[0].\ff{x}.\ff{get}(0)$ is the first element of it, if it's an array.

Atoms may have their $\lambda$~terms defined outside of \phic{} formal scope.
For example, the object at \lref{stdout} would be denoted as
\begin{equation}
\begin{split}
& \ff{stdout}(\ff{text}) \mapsto \lambda s.M_\text{stdout},
\end{split}
\end{equation}
where $M_\text{stdout}$ is a $\lambda$~term defined externally.

\subsection{Locators}

\begin{eodefinition}\label{def:locator}
Object \textbf{locator} is a unique dot-separated not-empty
collection of identifiers prepended by either $\xi$, $\rho$, or $\Phi$.
\end{eodefinition}

Locators are used to avoid ambiguity when referencing objects.
For example, the Eq.~\ref{eq:c-fin} may be refined as
\begin{equation}
\begin{split}
& \ff{c} \mapsto \llbracket \br
& \quad \ff{center} \mapsto \Phi.\ff{point}(\Phi.\ff{-3}, \Phi.\ff{9}), \br
& \quad \ff{radius} \mapsto \Phi.\ff{40}, \br
& \quad \varphi \mapsto \xi.\ff{center}, \br
& \quad \ff{is-inside}(\ff{p}) \mapsto \llbracket \br
& \quad \quad \varphi \mapsto \rho.\ff{distance}(\xi.\ff{p}).\ff{leq}( \br
& \quad \quad \quad \rho.\ff{radius} \br
& \quad \quad ) \br
& \quad \rrbracket \br
& \rrbracket,
\end{split}
\end{equation}
where $\xi$ denotes the current abstract object
and $\Phi$ refers to the anonymous abstract ``root'' object.
Defining an object in a global scope
means binding it to the object $\Phi$, unless it's an anonymous
object, as the one at \lrefs{succ}{succ-end}.

The most precise and complete formula for the object in the
Eq.~\ref{eq:c-fin2} would also include attribute names for
the object application:
\begin{equation}
\begin{split}
& \ff{c} \mapsto \llbracket \br
& \quad \ff{center} \mapsto \Phi.\ff{point}( \br
& \quad \quad \ff{x} \mapsto \Phi.\ff{-3}, \br
& \quad \quad \ff{y} \mapsto \Phi.\ff{9} \br
& \quad ), \br
& \quad \ff{radius} \mapsto \Phi.\ff{40}, \br
& \quad \varphi \mapsto \xi.\ff{center}, \br
& \quad \ff{is-inside}(\ff{p}) \mapsto \llbracket \br
& \quad \quad \varphi \mapsto \xi.\rho.\ff{distance}(\ff{to} \mapsto \xi.\ff{p}).\ff{leq}( \br
& \quad \quad \quad \ff{other} \mapsto \xi.\rho.\ff{radius} \br
& \quad \quad ) \br
& \quad \rrbracket \br
& \rrbracket.
\end{split}
\end{equation}



